\documentclass{article}
%\usepackage{polski} %to replace latex keywords in the document to Polish
\usepackage[utf8]{inputenc} %sets input encoding to UTF-8, needed for Polish, Japanese, etc.
\usepackage[T1]{fontenc} %needed for Polish characters
\usepackage{lmodern} %this font handles Polish characters properly
\usepackage[cm]{fullpage} %very small margins (around 1.5cm)
\usepackage{multicol} %use \begin{multicols}{#} for # columns
\usepackage{amssymb} %big bracets and other useful symbols
\usepackage{amsmath} %some mathematical symbols
\usepackage{enumitem} %remove vertical space in itemize with: [noitemsep,nolistsep]
\usepackage{url} %use \url{} in the document
%\usepackage{tipa} % \textpolhook{a}

\newcommand{\nodata}{\emph{no data yet}}
\newcommand{\samplecustom}[2]{X_{#1}, \ldots, X_{#2}}
\newcommand{\sample}{\samplecustom{1}{n}}

\newcommand{\distbernoulli}{\textrm{Bern}}
\newcommand{\distbinomial}{\textrm{Bin}}
\newcommand{\distpoisson}{\textrm{Pois}}
\newcommand{\distgeometric}{\textrm{G}}
\newcommand{\disthypergeometric}{\textrm{Hy}}
\newcommand{\distuniform}{\textrm{U}}
\newcommand{\distnormal}{\textrm{N}}
\newcommand{\distexponential}{\textrm{Exp}}
\newcommand{\distgamma}{\Gamma}
\newcommand{\distchisquare}{\chi^2}
\newcommand{\diststudentt}{\textrm{t}}
\newcommand{\distf}{\textrm{F}}

\newcommand{\param}{\theta}
\newcommand{\estim}{\hat{\theta}}
\newcommand{\limit}[2]{\underset{#1 \rightarrow #2}{\lim}}
\newcommand{\mean}{\overline{X}}
\newcommand{\quantile}[2]{\textrm{#1-quantile}_{#2}}
\newcommand{\qquantile}[1]{\quantile{q}{#1}}
\newcommand{\qqquantile}[1]{\textrm{quantile}_{#1}}
\newcommand{\med}{\textrm{MED}}
\newcommand{\mad}{\textrm{MAD}}
\newcommand{\var}{\textrm{Var}}
\newcommand{\statspace}{\left( \mathcal{H}, \mathcal{A}, \mathcal{P} = \left\{ p_\theta : \theta \in \Theta \right\} \right)}

\begin{document}

\title{Theory and formulas for \emph{Computer Statistics} course on MiNI on WUT}
\date{\today}
\author{Mateusz Bysiek, Computer Science, MiNI, WUT}
\maketitle

%\pagebreak[4]

\tableofcontents

\newpage

\section{Theory}

\subsection{Review of probability theory}

\subsubsection{Review of basic probability distributions}
\begin{multicols}{2}
\paragraph{\underline{Discrete}} \hspace{0pt}

\vspace{10pt} \noindent \textbf{Bernoulli}
$ X \sim \distbernoulli(p) $, $0 < p < 1$

$p$: probability of success
\begin{gather*}
P(X=1)=p \\ %\mbox{ , }
P(X=0)=1-p
\end{gather*}

$ \mathbb{E}X = p$, $\var X = p(1-p) $

\vspace{10pt} \noindent \textbf{Binomial}
$X \sim \distbinomial(n,p)$, $n \in \mathbb{N}_+$, $0 < p < 1$

$n$: number of tests/trials ($\distbinomial(1,p) \equiv \distbernoulli(p)$)

$p$: probability of success
\begin{gather*}
P(x=k) = \binom{n}{k} p^k(1-p)^{n-k}
\end{gather*}

$ \mathbb{E}X = np$, $\var X = np(1-p) $

\vspace{10pt} \noindent \textbf{Poison}
$X \sim \distpoisson(\lambda)$, $\lambda > 0$

In some sense, it's a generalization of the Binomial distribution.
\begin{gather*}
P(X=k) = \frac{\lambda^k}{k!}e^{-\lambda} \mbox{ , } k = 0,1,2,\ldots
\end{gather*}

$ \mathbb{E}X = \lambda$, $\var X = \lambda $

\vspace{10pt} \noindent \textbf{Geometric}
$X \sim \distgeometric(p)$
\begin{gather*}
P(X=k) = (1-p)^{k-1}p
\end{gather*}

if $k=1,2,\ldots$ then $\mathbb{E}X = \frac{1}{p}$, $\var X = \frac{1-p}{p^2}$

if $k=0,1,2,\ldots$ then $\mathbb{E}X = \frac{1-p}{p}$, $\var X = \frac{1-p}{p^2}$

\vspace{10pt} \noindent \textbf{Hypergeometric}
$X \sim \disthypergeometric(N,M,n)$
\begin{gather*}
P(X=k) \frac{ \binom{M}{k} \binom{N-M}{n-k} }{ \binom{N}{n} }
\end{gather*}

$ \max\{ 0, M-N+n \} \leq  k \leq \min\{ n, M \} $

\vfill
\columnbreak

\paragraph{\underline{Continuous}} \hspace{0pt}

\vspace{10pt} \noindent \textbf{Uniform}
$X \sim \distuniform[a,b]$, $a,b \in R$, $a < b$

$a$: lower limit

$b$: upper limit
\begin{gather*}
f(x) = \begin{cases}
\frac{1}{b-a} \mbox{ if } x \in [a,b] \\
0 \mbox{ otherwise}
\end{cases} 
= \frac{1}{b-a} I_{[a,b]}(x)
\end{gather*}

$ \mathbb{E}X = \frac{a+b}{2}$, $\var X = \frac{(b-a)^2}{12} $

\vspace{10pt} \noindent \textbf{Normal}
$X \sim \distnormal(\mu, \sigma)$

$\mu$: mean

$\sigma$: standard deviation
\begin{gather*}
f(x) = \frac{1}{\sqrt{2\pi} \sigma} \exp{ \left\{ - \frac{(X-\mu)^2}{2 \sigma^2} \right\} }
\end{gather*}

$ \mathbb{E}X = \mu$, $\var X = \sigma^2$

\vspace{10pt} \noindent \textbf{Exponential}
$X \sim \distexponential(\lambda)$, $\lambda > 0$

decreasing \eqref{eq:expdecreasing} and increasing \eqref{eq:expincreasing}:
\begin{gather*}
f(x) = \lambda e^{-\lambda x}
\label{eq:expdecreasing} \tag{a}
\end{gather*}

\vspace{-30pt} \begin{gather*}
f(x) = 1 - e^{-\lambda x}
\label{eq:expincreasing} \tag{b}
\end{gather*}

\vspace{10pt} \noindent \textbf{Gamma}
$X \sim \distgamma(\alpha, \beta)$, $\alpha, \beta > 0$

\nodata

\vspace{10pt} \noindent \textbf{Chi-square}
$X \sim \distchisquare_n$

$n$: number of degrees of freedom

\vspace{10pt} \noindent \textbf{Student's t-distribution}
$X \sim \diststudentt^{[n]}$

$n$: number of degrees of freedom

Also known as simply t-distribution.

\vspace{10pt} \noindent \textbf{F-distribution}
$X \sim \distf^{[n]}$

$n$: number of degrees of freedom

Also known as Fisher-Snedecor distribution.

\end{multicols}

\newpage

\subsection{Descriptive statistics}

\subsubsection{Basic notions}
\paragraph{Population} is a set, of which we would like to learn some properties.
  
\paragraph{Sample} is a subset of a population, on which we calculate the properties and then make
statements about the whole population.

\paragraph{Random sample} is a sample taken in an unpredictable way, i.e. the method of taking the
sample involves an unpredictable component.

\paragraph{Self-weighting sample} is a sample taken in such manner that each element of population has equal
chance of being included in the sample.

\paragraph{Simple random sample} is a kind of self-weighting sample selected so that all samples of
same size have equal chance of being selected.

\subsection*{Remark}
Further on we in most cases assume random sample of size $n$, $X = \sample$.


\newpage

\begin{multicols}{2}

\subsubsection{Measures of location: central tendency}
\paragraph{Mean} \hspace{0pt}

\vspace{-20pt}
\begin{gather*}
\mean = \frac{1}{n}\sum_{i=1}^n X_i
\end{gather*}

\paragraph{Mean for grouped data} \hspace{0pt}

\vspace{-5pt}
\begin{gather*}
\mean = \frac{1}{n}\sum_{i=1}^n n_i x_i
\end{gather*}

where $n_i$ is count of $i$th result values, and $x_i$ is the $i$th result value.

%\paragraph{Weighted mean} \nodata

\paragraph{Median} \hspace{0pt}

\vspace{-20pt}
\begin{gather*}
\med = \frac{1}{2} \left( X_{\lfloor \frac{n+1}{2} \rfloor : n} + X_{\lceil \frac{n+1}{2} \rceil : n} \right) =\\ 
= \begin{cases}
 X_{\frac{n+1}{2} : n} \mbox{ if } n \mbox{ is odd} \\
 \frac{1}{2} \left( X_{\frac{n}{2} : n} + X_{\frac{n+2}{2} : n} \right) \mbox{ if } n \mbox{ is even}
\end{cases}
\end{gather*}

where $x_{k:n}$ is the $k$th sample element taken from sorted sample.

\paragraph{Weighted median}
\begin{gather*}
\med = X_{\med}^L + \frac{b}{n_{\med}} \left( \frac{n}{2} - \sum_{i=1}^{i_{\med}-1} n_i \right)
\end{gather*}

\paragraph{Mode}
is an observation that appears most often.

\paragraph{Trimmed mean} \hspace{0pt}

\vspace{-5pt}
\begin{gather*}
\mean_{t,k} = \frac{1}{n - 2k} \sum_{i=k+1}^{n-k} X_i
\end{gather*}

\paragraph{Winsorized mean}
is thought to decrease the effect of outliers.
\begin{gather*}
\mean_{w,k} = \frac{1}{n} \left[ (k+1)X_{k+1} + \sum_{i=k+2}^{n-k-1} + (k+1)X_{n-k} \right]
\end{gather*}

\paragraph{Geometrical mean}

\[ \mean_g = \sqrt[n]{x_1 x_2 \ldots x_n} \]

\paragraph{Harmonic mean} \hspace{0pt}

\vspace{-10pt}
\[ \mean_h = \frac{1}{ \frac{1}{n} \sum_{i=1}^n \frac{1}{X_i} } \]


\vfill
\columnbreak

\subsubsection{Measures of location: position}
\paragraph{Quantiles}
are points taken at regular intervals from the cumulative distribution function.

% \noindent
Quantiles have two equivalent notations:
\begin{itemize}[noitemsep,nolistsep]

  \item Not parametrized, $p$th quantile is equivalent to argument value for which c.d.f. has value
  $p$, where $p \in \mathbb{R} \cap [0,1]$.

  \item Parametrized with number of intervals equal $q$, these are called q-quantiles. For each $q$, there
  are $q-1$ q-quantiles: $\qquantile{1}$, $\qquantile{2}$, \ldots , $\qquantile{q}$.

\end{itemize}

\vspace{-10pt}
\begin{gather*}
\quantile{q}{k} \equiv \qqquantile{k/q}
\end{gather*}

Quantiles of Standard Normal distribution ($\distnormal(0,1)$) are denoted by $z$, for example: $z_{0.5}$ is a median of $\distnormal(0,1)$ 

\paragraph{Quartiles}
are 4-quantiles. They are denoted as follows: 

$Q_1 \equiv \quantile{4}{1} \equiv \qqquantile{0.25}$

$Q_2 \equiv \quantile{4}{2} \equiv \qqquantile{0.5} \equiv \med$

$Q_3 \equiv \quantile{4}{3} \equiv \qqquantile{0.75}$

\paragraph{Deciles}
are 10-quantiles, denoted by $D_1, \ldots , D_9$.

\paragraph{Percentiles}
are 100-quantiles, denoted by $P_{1}, \ldots , P_{99}$.

% \begin{align*}
% \textrm{Median: } & Med = 
% \left\{ \begin{array}{cc}
% \textrm{odd }n\textrm{: } & x_{\frac{n+1}{2}} \\ %(n+1)/2  n/2  n/2 + 1
% \textrm{even }n\textrm{: } & \frac{1}{2} \left( x_{\frac{n}{2} } + x_{\frac{n}{2}+1} \right)
% \end{array} \right\} \\
% \textrm{Mode: } & \textrm{observation that appears most often} \\
% \textrm{Quantiles: } & \ldots
% \end{align*}

% \subsubsection{Means}
% \begin{align*}
% \textrm{Average: } & \bar{x} = \frac{1}{n}\sum_{i=1}^n x_i \\
% \textrm{Trimmed: } & \sum_{i=1}^n \ldots \\
% \textrm{Windsorized: } & \sum_{i=1}^n \ldots \\
% \textrm{Geometrical: } & \ldots \\
% \textrm{Harmonic: } & \ldots
% \end{align*}


\subsubsection{Measures of dispersion}
\paragraph{Range} is very prone to outliers.
$ R = \max(X) - \min(X) $

\paragraph{Interquantile range}
$ IQR = Q_3 - Q_1 $

\paragraph{Variance} \hspace{0pt}

\vspace{-30pt}
\begin{gather*}
\var X = s^2 = \frac{1}{n-1}\sum_{i=1}^n \left( X_i - \mean \right)^2
\end{gather*}

\paragraph{Standard deviation}
$s = \sqrt{s^2} = \sqrt{Var}$

\paragraph{Coefficient of variation} \hspace{0pt}

\vspace{-20pt}
\begin{gather*}
cv = \frac{s}{\mean}
\end{gather*}

\paragraph{Median absolute deviation} \hspace{0pt}

\vspace{-20pt}
\begin{gather*}
\mad = \med \left( |X_1 - \med_X|, \ldots, |X_n - \med_X| \right)
\end{gather*}

\vspace{-10pt} %\noindent
where $\med_X$ is the sample median.


\subsubsection{Measures of shape}
\paragraph{Skewness} indicates asymmetry. There are 3 cases:
\begin{enumerate}[noitemsep,nolistsep]
  \item $sk = 0$ the sample is symmetric
  \item $sk > 0$ the sample is heavier on the right
  \item $sk < 0$ the sample is heavier on the left
\end{enumerate}

\noindent The direction of inequality symbol points where the tail is.

\paragraph{Kurtosis} or so-called ``peakedness''.


\end{multicols}

\newpage

\subsection{Inferential statistics}

\subsubsection{Basics, more on probability distributions}

\begin{multicols}{2}
\paragraph{Empirical distribution function} (shortly e.d.f.)
based on a sample $X = \sample$ is a function:
%Empirical distribution function:
%\[ \hat{F_n}(t; X_1, \ldots , X_n) = \frac{1}{n} \Sigma_{i=1}^n I_{(-\infty,t]} (X_i)
%\; , \; t \in R \]
\[ \hat{F}_n(x) = \frac{\#\{i: X_i \leq x\}}{n} \mbox{ or, defined in other way:} \]
\[ \hat{F}_n(x) = \frac{1}{n} \sum_{i=1}^{n} I_{(-\infty,x]} (X_i) 
\mbox{ , where }
I_n(x) = \begin{cases} 1 \mbox{ if } x \in A \\ 0 \mbox{ otherwise} \end{cases} \]

The e.d.f. is a cumulative distribution function based not no the whole population, but only the sample.

\paragraph{Glivenko cantelli lemma}

Let $\sample$ i.i.d. $F$ and let $\hat{F}_n$ denote the empirical distribution
function based on this sample. Then:
\[  P ( \lim_{n \rightarrow \infty}  \sup_{x \in R} | \hat{F}_n(x) - F(x) | = 0 ) = 1 \]

This lemma tells us that when sample size is tending to infinity, then the maximum difference
between e.d.f. and cumulative d.f. of the whole population will tend to zero.

\paragraph{Statistical space}
\[  \statspace \]
\noindent where:
\begin{itemize}[noitemsep,nolistsep]
  \item $\mathcal{H}$ is sample space
  
  \item $\mathcal{A}$ is $\sigma$-algebra of $\mathcal{H}$
  
  \item $\mathcal{P}$ is a family of distributions
  
  \item $\theta$ is value of the parameter
  
  \item $\Theta$ is family of parameters
\end{itemize}

\paragraph{Sufficient statistic}
is a such statistic that is guaranteed to get us as much information as any statistic based on at
most the same sample size.

A statistic is said to be sufficient for the parameter $\param$ (for a family $\mathcal{P}$) iff the
conditional probability (given below) does not depend on $\param$ for any value $t$ on statistic
$T$.
\begin{gather*}
\Pr(X=x|T(X)=t,\theta) = \Pr(X=x|T(X)=t)\\
\Pr(x|t,\theta) = \Pr(x|t)
\end{gather*}

%\[ \]

\paragraph{Factorization criteria}
states that a statistic $T$ is sufficient iff $\exists$ the following factorization:
\[ p_\param(\sample) = g_\param(T(\sample)) h(\sample) \]
\noindent where:
\begin{itemize}[noitemsep,nolistsep]
  \item $p_\param$ is a distribution under study
  
  \item $g$ depends on $\param$ and a sample but only through $T$
  
  \item $h$ is independent of $\theta$
\end{itemize}

\paragraph{Family of k-parametric exponential distributions}
contains those distributions that satisfy the following:
\[ p_\theta(x) = h(x) \exp{ \left\{ \sum_{j=1}^k C_j(\theta)S_j(x) + B(\theta) \right\}  } \]

where $S_1, \ldots, S_k$ are linearly independent.

\end{multicols}

\newpage

\subsubsection{Point estimation}

\begin{multicols}{2}
\paragraph{Estimator} $\estim$ is a function that estimates value of an unknown parameter $\param$.

\paragraph{Bias of an estimator}
$ b(\estim) = $

$ = \mathbb{E}(\estim(\sample)-\param) = \mathbb{E}(\estim)-\param $
\begin{equation*}
b(\estim) = \begin{cases}
\left.\begin{aligned}
   > 0 \rightarrow \mbox{overestimation} \hspace{7pt} \\
   < 0 \rightarrow \mbox{underestimation} 
\end{aligned}
\right\} \mbox{ systematic error} \\
\hspace{3pt} = 0 \rightarrow \mbox{no systematic error} 
\end{cases}
\end{equation*}

\paragraph{Unbiased estimator}
$ \mathbb{E} \hat\theta(X_1, \ldots, X_n) = \theta $

\paragraph{Asymptotically unbiased estimator}
is an estimator that is unbiased when sample size tends to infinity.

$ \underset{n \rightarrow \infty}{\lim} b(\hat{\theta}(X_1, \ldots, X_n)) = 0 $

\paragraph{Mean squared error}
\[ MSE(\hat{\theta}) = \mathbb{E}(\hat{\theta}-\theta)^2 
= Var(\hat{\theta})+b^2(\hat{\theta}) \]

\paragraph{Regular model} is a stat. space $ \statspace $ that satisfies the following conditions:

\begin{itemize}[noitemsep,nolistsep]
  \item support of $p_\theta$ does not depend on $\theta$; 
  
  support of $ p_\theta(x) = \left\{ x: p_\theta(x) > 0 \right\} $
  
  \item $ 0 < \mathbb{E} \left[ \frac{\partial}{\partial\theta} \ln p_\theta(x) \right]^2 < \infty $
\end{itemize}

\paragraph{Fisher's information contained in a single observation}
\[ I(\theta) = \mathbb{E} \left[ \frac{\partial}{\partial\theta} \ln p_\theta(x) \right]^2  \]

\paragraph{Fisher's information contained in a sample}
\[ I_n(\theta) = \mathbb{E} \left[ \frac{\partial}{\partial\theta} \ln p_\theta(x_1,\ldots,x_n) \right]^2  \]

\paragraph{Cramer-Rao inequality}
$X = X_1,\ldots,X_n$ is from regular model. Let $T$ denote
an unbiased estimator of a function $g(\theta)$, then:
\[ Var T(X) \geq \frac{\left( g'( \theta ) \right)^2}{n I(\theta)} \]

\paragraph{Fisher's information inequality}
states that if $\theta$ is an unbiased estimator, then:
\[ Var\hat\theta(X) \geq \frac{1}{n I(\theta)} \]

\paragraph{Efficiency of an estimator}
\[ eff(\hat\theta) = \frac{1}{Var(\hat\theta) n I(\theta)} \]
and $0 < eff(\hat\theta) \leq 1$

\paragraph{Most efficient estimator}
has $eff(\hat\theta) = 1$

\paragraph{Asymptotically efficient estimator}
is not efficient but $\underset{n \rightarrow \infty}{\lim} eff(\hat{\theta}(\sample)) = 1 $

\paragraph{Relative efficiency}
\[ \frac{eff(\estim_1)}{eff(\estim_2)} = \frac{Var(\estim_2)}{Var(\estim_1)} \]

\paragraph{Consistency of an estimator}
is preserved, if, $\forall \mathcal{E} > 0$ for a given estimator $\hat{\theta}$ the following holds:

\[ \limit{n}{\infty}
 P \left( \left| \estim(\sample) - \theta \right| < \mathcal{E} \right) = 1 \]

Also, if $\estim$ is unbiased estimator of $\param$
and if $ \limit{n}{\infty} Var(\estim(\sample)) = 0 $ then $\estim$ is consistent.

\paragraph{Differentiation of estimators}
by the method of construction:

\begin{enumerate}[noitemsep,nolistsep]
  \item method-of-moments estimator: MME
  
  \item maximal-likelihood estimator: MLE
  
  \item minimum-variance unbiased estimator MVUE
  
  \item method-of-quantiles estimator MQE
\end{enumerate}

\paragraph{MME}
- method of moments.

$ \param = g(\mathbb{E}X, \mathbb{E}X^2, \ldots, \mathbb{E}X^r) $

$ \estim = g(\mu_1,\mu_2,\ldots,\mu_r) $

where $\mu_k = \frac{1}{n} \sum_{i=1}^n X_i^k$ e.g. $\mu_1 = \bar{X}$

\paragraph{MLE}
- method of maximal likelihood.

We use likelihood function: $L = L(\param, \sample) = $
\[ = \begin{cases}
P_\theta(X_1 = x_1, \ldots, X_n = x_n) \mbox{ for discrete distr.} \\
f_\theta(x_1, \ldots, x_n) \mbox{ for cont.}
\end{cases} \]

$ \estim = \arg \max L(\param) $

If $\estim$ denotes a M.L.E. of $\param$ in the regular model, then

\[ \left( \estim(\sample) - \theta \right) \sqrt{n}
 \underset{n \rightarrow \infty}{\sim}  \distnormal\left( 0, \frac{1}{\sqrt{I(\param)}} \right) \]

and for large $n$:

\[ \estim \sim \distnormal\left( 0, \frac{1}{\sqrt{n I(\param)}} \right) \]

% \paragraph{Uniform distribution estimators}
% for $\theta$ in $U([0,\theta])$ are:
% \begin{align*}
% \textrm{MME: } & \hat{\theta}_1 = 2\bar{X} \\ 
% \textrm{MLE: } & \hat{\theta}_2 = X_{n:n} \\ 
% \textrm{MVUE: } & \hat{\theta}_3 = \frac{n+1}{n}X_{n:n} = \frac{n+1}{n}\hat{\theta}_2 \\ 
% %\textrm{MQE: } & ?
% \end{align*}

% \subsubsection{Normal distribution}
% \begin{align*}
% \textrm{MME: } & \\ 
% \textrm{MLE: } & \\ 
% \textrm{MVUE: } & \\ 
% \textrm{MQE: } & 
% \end{align*}


\end{multicols}

\newpage

\subsubsection{Interval estimation}

\begin{multicols}{2}
\paragraph{Confidence interval}
is a random interval, within which a true value of an estimated parameter $\param$ lies. We cannot say
where exactly inside the interval the above-mentioned value is placed.
\[ P(T_L < \param < T_U) \geq 1 - \alpha \]

\paragraph{Confidence interval length}
is defined as $|T_U - T_L|$.

\vfill

\paragraph{Confidence level}
is defined as $ 1 - \alpha $ or $ (1 - \alpha)100\% $, depending on what units we want to use.
$\alpha$ is significance level, described later on.

\paragraph{Confidence interval properties}
with sample size $n$, confidence interval length $l = |T_U - T_L|$, confidence level $c = 1 - \alpha$

\vspace{10pt} \noindent are as follows:
\begin{itemize}[noitemsep,nolistsep]
  \item $c \nearrow \; \Rightarrow \; l \nearrow$
  
  \item for fixed $c$: sample size $n \nearrow \; \Rightarrow \; l \searrow$
  
  %\item coś jeszcze?
  
\end{itemize}

\end{multicols}

\subsubsection{Selected confidence intervals}

\begin{multicols}{3}
\paragraph{\underline{Conf. intervals for the mean}}
$\mu$ \hspace{0pt} \newline
Conf. interval: $ \left( \mean - k, \mean + k \right) $

\vspace{5pt} \noindent \textbf{\em Model 1}:
$X$ i.i.d. $\distnormal(\mu, \sigma)$, known $\sigma$
\[ k = z_{1-\alpha/2}\frac{\sigma}{\sqrt{n}} \]

\noindent \textbf{\em Model 2}:
$X$ i.i.d. $\distnormal(\mu, \sigma)$, $\sigma=?$
\[ k = \diststudentt^{[n-1]}_{1-\alpha/2}\frac{s_X}{\sqrt{n}} \]

\noindent \textbf{\em Model 3}:
$X$ i.i.d. ?, large $n$
\[ k = z_{1-\alpha/2}\frac{s_X}{\sqrt{n}} \]

\paragraph{\underline{Conf. intervals for variance}}
$\sigma^2$ \hspace{0pt} \newline
or $\var X$ are, in case of first 2 models, defined as:
\[ \left( f(1-\alpha/2), f(\alpha/2) \right) \]
and $f(k)$ is defined for each. The third model has its own definition.

\vspace{5pt} \noindent \textbf{\em Model 1}:
$X$ i.i.d. $\distnormal(\mu, \sigma)$, known $\mu$
\[ f(k) = \frac{n\tilde{s}_X^2}{\distchisquare_{k,n}} \]

\noindent \textbf{\em Model 2}:
$X$ i.i.d. $\distnormal(\mu, \sigma)$, $\mu=?$
\[ f(k) = \frac{(n-1)\tilde{s}_X^2}{\distchisquare_{k,n-1}} \]

\noindent \textbf{\em Model 3}:
$X$ i.i.d. ?, large $n$,

Conf. interval: $ \left( f(-1) , f(1) \right) $
\[ f(k) = \frac{(2n-2)\tilde{s}_X^2}{\sqrt{2n-3} + (k)z_{1-\alpha/2}} \]

\paragraph{\underline{Conf. interval for proportion}}
$p$ \hspace{0pt} \newline
i.e. probability of success.

\vspace{5pt} \noindent \textbf{\em Model 1}:
$X$ i.i.d. $\distbernoulli(p)$, large $n$

Conf. interval: $ \left( \hat{p}-l , \hat{p}+l \right) $
\[ \hat{p} = \frac{1}{n} \sum_{i=1}^n X_i \mbox{ and }
l = z_{1-\frac{\alpha}{2}}\sqrt{\frac{\hat{p}(1-\hat{p})}{n}} \]

\end{multicols}

\subsubsection{Planning experiments}

\begin{multicols}{2}
\paragraph{Two stage sampling procedure}
called Stein's sampling procedure:
\begin{enumerate}
  \item take a preliminary sample of small size $n_0$
  
  $x_1, \ldots, x_{n_0} \rightarrow X_0 \ \longrightarrow \ s^2_0 \rightarrow s_0$
  
  \item estimate how many more measurements you need:
  \begin{gather*}
  n = \lceil \left( \diststudentt^{[n_0-1]}_{1-\alpha/2} \frac{s_0}{d} \right) \rceil
  \end{gather*}
  if $n > n_0$ then take the second sample of size $n-n_0$
  
\end{enumerate}

\end{multicols}

\newpage

\subsubsection{Hypothesis testing}

\begin{multicols}{2}
\paragraph{Statistical test}
for testing hypothesis $H$ against $K$, defined on sample $\sample$, is a function:
\begin{gather*}
\psi: \mathcal{H} \rightarrow \{0, 1\}
\end{gather*}
where $\psi = 1$ means that we reject $H$ (i.e. we accept $K$) and $\psi = 0$ means that we do not reject $H$.

Usually, $\psi$ has the following form:
\begin{gather*}
\psi(\sample) = 
\begin{cases}
1 \mbox{ if } T(\sample) \in \mathcal{K}_\alpha \\
0 \mbox{ if } T(\sample) \not\in \mathcal{K}_\alpha
\end{cases}
\end{gather*}
where $T$ is a test statistic, and $\mathcal{K}_\alpha$ is a critical region.

\paragraph{Null hypothesis}
is usually denoted by $H$, and it is a hypothesis that we test. In most of the cases it is some
phenomenon about which we are not sure whether it really happens, and we perform a test in order to
determine the reality.

\paragraph{Alternative hypothesis}
is usually denoted by $K$, and it is a hypothesis opposite to the null hypothesis.

\paragraph{Critical region}
is a set of all outcomes of a test, for which we reject the null hypothesis. Denoted by $\mathcal{K}_\alpha$.

\paragraph{Type I error}
is made when $H$ is rejected when in fact it is true. Probability of this error is denoted by $\alpha_\psi$.

\paragraph{Type II error}
is made when $H$ is not rejected when in fact it is false. Probability of this error is denoted by $\beta_\psi$.

\paragraph{Significance level}
denoted with $\alpha$ is the upper bound for probability of making type I error. It is fixed in
advance.
\begin{gather*}
\begin{cases}
\alpha_\psi \leq \alpha \\
\beta_\psi \rightarrow \min
\end{cases} \equiv 
\begin{cases}
P(\psi = 1 | H) \leq \alpha \\
P(\psi = 0 | K) \rightarrow \min
\end{cases}
\end{gather*}

``Significance'' in statistics does not have the usual everyday meaning.

\paragraph{Power of a test}
is a probability of rejecting false hypothesis.
\begin{gather*}
\Pi = P(\psi = 1 | K) = 1 - P(\psi = 0 | K) = 1 - \beta_\psi
\end{gather*}

\paragraph{Uniformly most powerful test}
(or U.M.P.T.) is such test $\psi^*$, defined for hypothesis $H: \theta \in \Theta_H$ against $K: \theta \in
\Theta_K$, on the significance level $\alpha$, that satisfies the following:
\begin{gather*}
\forall \theta \in \Theta_K : \mathbb{E}_\theta \psi^*(X) \geq \mathbb{E}_\theta \psi(X)
\end{gather*}

\paragraph{Neyman-Pearson Lemma}
states, that if we consider a testing problem $H: f_0$ against $K: f_1$, defined as follows:

Let $\alpha \in (0,1)$ and let $\psi$ denote a test that:
\begin{gather*}
\psi(X) = \begin{cases}
1 \mbox{ if } f_1(X) > k f_0(X) \\
0 \mbox{ if } f_1(X) < k f_0(X)
\end{cases}
\label{eq:neymanpearson1} \tag{A}
\end{gather*}

where $k$ is a constant, $k > 0$, and it satisfies:
\begin{gather*}
\mathbb{E}_{f_0} \psi(X) = \alpha
\label{eq:neymanpearson2} \tag{B}
\end{gather*}

\noindent \ldots then we may conclude the following:
\begin{enumerate}

  \item Test $\psi$ satisfying \eqref{eq:neymanpearson1} and \eqref{eq:neymanpearson2} is the
  U.M.P.T. for $H$ against $K$ on the significance level $\alpha$.

  \item If $\alpha$ is the U.M.P.T. for $H$ against $K$ on the significance level $\alpha$, the test
  $\psi$ should satisfy \eqref{eq:neymanpearson1} and \eqref{eq:neymanpearson2}.

\end{enumerate}

\paragraph{Monotonic likelihood ratio}
for a family $P = \{ f_\theta : \theta \in \Theta \}$ exists if there is a statistic $T$
such that $\frac{ p_{\theta_1}(X) }{ p_{\theta_0}(X) }$ for $\theta_1 > \theta_0$ is a nondecreasing
function of $T(X)$.

\paragraph{Karlin-Rubin theorem}
let $P = \left\{ f_\theta : \theta \in \Theta \right\}$ denote a family with the monotonic
likelihood ratio with respect to statistic $T$. Consider the hypothesis $H: \theta \leq \theta_0$
against $K: \theta > \theta_0$. Then, for any $\alpha \in (0,1)$ there exists the U.M.P.T on the
significance level $\alpha$ of the form:
\begin{gather*}
\psi(\sample) = \begin{cases}
1 \mbox{ if } T(\sample) > C_\alpha \\
0 \mbox{ otherwise}
\end{cases}
\end{gather*}

\paragraph{Unbiasedness of a hypothesis test}
is determined by checking if for a test $\psi$ the following holds:
\begin{gather*}
\begin{cases}
\forall \theta \in \Theta_H : \mathbb{E}_\theta \psi(X) \leq \alpha \\
\forall \theta \in \Theta_K : \mathbb{E}_\theta \psi(X) \geq \alpha
\end{cases}
\end{gather*}

\paragraph{Consistent hypothesis test}
is such test for H against K, for which the following holds:
\begin{gather*}
\forall \theta \in \Theta_K \limit{n}{\infty} \mathbb{E}_\theta \psi(\sample) = 1
\end{gather*}

\paragraph{Two categories}
of hypothesis tests are: parametric and non-parametric.

\paragraph{Parametric test}
is a test, in which we have parameters that uniquely define the distribution to which the sample
under study belongs.

\paragraph{Non-parametric statistical test}
is a test that does not assume that sample belongs to any specific distribution. Such test is also
called a distribution-free test.

\paragraph{p-value}
is an indicator for rejection or acceptance of null hypothesis. When it is not smaller than
significance level of the test, we accept our hypothesis.

\end{multicols}

\newpage

\subsubsection{Selected parametric tests}

\begin{multicols}{2}
\paragraph{\underline{Hypothesis tests for the mean}}
$\mu$

\vspace{5pt}
\noindent \textbf{One-sample z-test}
$H_0: \mu=\mu_0$

We use it to check hypothesis that mean $\mu$ is equal to the specific value $\mu_0$. Critical
region is: $ \mathcal{K}_\alpha = (-\infty, k] \cup [k, +\infty) $

\vspace{5pt}
\textbf{\em Model 1}:
$X$ i.i.d. $\distnormal(\mu, \sigma)$, known $\sigma$

\vspace{-15pt}
\begin{gather*}
T=\frac{\bar{X}-\mu_0}{\sigma}\sqrt{n} 
\overset{H_0}{\longrightarrow} \distnormal(0,1)
\end{gather*}

\vspace{-20pt}
\begin{gather*}
k = z_{1-\frac{\alpha}{2}}
\end{gather*}

\textbf{\em Model 2}:
$X$ i.i.d. ?, large $n$

\vspace{-15pt}
\begin{gather*}
T=\frac{\bar{X}-\mu_0}{s_X}\sqrt{n} 
\overset{H_0}{\underset{n \rightarrow \infty}{\longrightarrow}} \distnormal(0,1)
\end{gather*}

\vspace{-15pt}
\begin{gather*}
k = \diststudentt^{[n-1]}_{1-\frac{\alpha}{2}}
\end{gather*}

\noindent \textbf{One-sample t-test}
$X$ i.i.d. $\distnormal(\mu, \sigma)$, $\sigma=?$, $H_0: \mu=\mu_0$

\vspace{-15pt}
\begin{gather*}
z=\frac{\bar{X}-\mu_0}{s_X}\sqrt{n} 
\overset{H_0}{\longrightarrow} \diststudentt^{[n-1]}
\end{gather*}

\vspace{-15pt}
\begin{gather*}
k = z_{1-\frac{\alpha}{2}}
\end{gather*}

\noindent \textbf{Two-sample z-test}

We use it to check a hypothesis that means of two samples are equal. The test can be applied in
following 2 cases:

\vspace{10pt}
\textbf{\em Model 1}:
$X$, $Y$ i.i.d. $\distnormal(\mu_{X,Y}, \sigma_{X,Y})$,

known $\sigma_X$ and $\sigma_Y$, $H_0: \mu_X=\mu_Y$

\vspace{-15pt}
\begin{gather*}
a_I=\frac{\sigma_I^2}{n_I} \;,\;
z=\frac{\bar{X}-\bar{Y}}{ \sqrt{a_X+a_Y} }
\longrightarrow \distnormal(0,1)
\end{gather*}

\textbf{\em Model 2}:
$X$, $Y$ i.i.d. ?, large $n_{X,Y}$,

$H_0: \mu_X=\mu_Y$

\vspace{-15pt}
\begin{gather*}
a_I=\frac{s_X^2}{n_X} \;,\;
z=\frac{\bar{X}-\bar{Y}}{ \sqrt{a_X+a_Y} }
\underset{n \rightarrow \infty}{\longrightarrow} \distnormal(0,1)
\end{gather*}

\noindent \textbf{Two-sample t-test}

\vspace{10pt}
\textbf{\em Model 1}:
$X$, $Y$ i.i.d. $\distnormal(\mu_{X,Y}, \sigma_{X,Y})$, 

$\sigma_X=\sigma_Y=?$, $H_0: \mu_X=\mu_Y$, 
$d = n_X+n_Y-2$, $a_i = (n_i - 1)s_i^2$

\vspace{-15pt}
\begin{gather*}
\dot{s}^2 = \frac{a_X + a_Y}{d} \;,\; 
b_I=\frac{\dot{s}^2}{n_I}
\end{gather*}

\[ t=\frac{\bar{X}-\bar{Y}}{ \sqrt{b_X+b_Y} }
\overset{H_0}{\longrightarrow} \diststudentt^{[d]} \]

\textbf{\em Model 2}:
$X$, $Y$ i.i.d. $N(\mu_{X,Y}, \sigma_{X,Y})$, 

$\sigma_X=?$ and $\sigma_Y=?$, $H_0: \mu_X=\mu_Y$

\vspace{-15pt}
\begin{gather*}
a_i = \frac{s_i^2}{n_i} \;,\;
d = \frac{(a_X + a_Y)^2}{ \frac{a_X^2}{n_X-1}+\frac{a_Y^2}{n_Y-1} }
\end{gather*}

\vspace{-15pt}
\begin{gather*}
t=\frac{\bar{X}-\bar{Y}}{ \sqrt{ a_X + a_Y } }
\overset{H_0}{\longrightarrow} \diststudentt^{[d]}
\end{gather*}

\vfill
\columnbreak

\noindent \textbf{Paired t-test}
$X$, $Y$ i.i.d. $\distnormal(\mu_{X,Y}, \sigma_{X,Y})$, 

$\sigma_X=?$ and $\sigma_Y=?$, $H_0: \mu_X=\mu_Y$, 

$\forall i: (X_i,Y_i)$ are dep., $Z=X-Y$

\vspace{-15pt}
\begin{gather*}
t= \frac{\bar{Z}-0}{s_Z}\sqrt{n} 
\overset{H_0}{\longrightarrow} \diststudentt^{[n-1]}
\end{gather*}

It is usually used when identical samples are taken at two points in time, and we wish to check
whether there are significant differences between them.

\paragraph{\underline{Hypothesis tests for variance}}
$\sigma^2$

\vspace{5pt}
\noindent \textbf{One-sample test}
$X$ i.i.d. $\distnormal(\mu, \sigma)$, $H_0: \sigma^2=\sigma_0^2$

We use this test to test a hypothesis that $\sigma^2$ is equal to a specific value $\sigma_0^2$.
$ \mathcal{K}_\alpha = (0, \distchisquare_{k, \frac{\alpha}{2}}] \cup [\distchisquare_{k, 1-\frac{\alpha}{2}}, +\infty) $

\vspace{5pt}
\textbf{\em Model 1}:
known $\mu$

\vspace{-15pt}
\begin{gather*}
T = \frac{n s_X^2}{\sigma_0^2}
\overset{H_0}{\longrightarrow} \distchisquare_{n-1}
\end{gather*}

\vspace{-15pt}
\begin{gather*}
k = n
\end{gather*}

\textbf{\em Model 2}:
$\mu=?$

\vspace{-15pt}
\begin{gather*}
T = \frac{(n-1)s_X^2}{\sigma_0^2}
\overset{H_0}{\longrightarrow} \distchisquare_{n-1}
\end{gather*}

\vspace{-15pt}
\begin{gather*}
k = n-1
\end{gather*}

\noindent \textbf{F-test}
$X$, $Y$ i.i.d. $\distnormal(\mu_{X,Y}, \sigma_{X,Y})$, 

$\mu_X=?$ and $\mu_Y=?$, $H_0: \sigma_X^2=\sigma_Y^2$

We use this test to test a hypothesis that two variances $\sigma_X^2$ and $\sigma_Y^2$ (for samples
$X$ and $Y$ respectively) are equal.

\vspace{-15pt}
\begin{gather*}
F = \frac{s_X^2}{s_Y^2} \overset{H_0}{\longrightarrow} \distf^{[n_1-1,n_2-2]}
\end{gather*}

\paragraph{\underline{Hypothesis tests for proportion}} $p$

\vspace{5pt}
\noindent \textbf{One-sample z-test}
$X$ i.i.d. $Bern(p)$, $H_0: p=p_0$

We use this test to test a hypothesis that $p$ is equal to a specific value $p_0$. It applies to two
such models:

\vspace{10pt}
\textbf{\em Model 1}:
$n_{X,Y}\geq100$

\vspace{-15pt}
\begin{gather*}
z=\frac{\hat{p}-p_0}{\sqrt{p_0(1-p_0)}}\sqrt{n}
\overset{H_0}{\underset{n \rightarrow \infty}{\longrightarrow}} \distnormal(0,1)
\end{gather*}

\textbf{\em Model 2}:
$n_{X,Y}<100$, $A = \arcsin$

\vspace{-15pt}
\begin{gather*}
z=2\left(A{\sqrt{\hat{p}}}-A{\sqrt{p_0}} \right)\sqrt{n}
\longrightarrow \distnormal(0,1)
\end{gather*}

\noindent \textbf{Two-sample z-test}
$X$, $Y$ i.i.d. $\distbernoulli(p_{X,Y})$, $H_0: p_X=p_Y$

We use this test to test a hypothesis that parameters $p_X$ and $p_Y$ (which are empirical success
rates for samples $X$ and $Y$ respectively) are equal. It applies to two such models:

\vspace{10pt}
\textbf{\em Model 1}:
$n_{X,Y}\geq100$, $ k_I = \Sigma_{i=1}^{n_I} I_i $

$ \dot{p}=\frac{k_X+k_Y}{n_X+n_Y} \mbox{ , } \dot{n}=\frac{n_X n_Y}{n_X+n_Y} $

\vspace{-15pt}
\begin{gather*}
z=\frac{\hat{p_X}-\hat{p_Y}}{\sqrt{\dot{p}(1-\dot{p})}}\sqrt{\dot{n}}
\underset{n_{X,Y} \rightarrow \infty}{\longrightarrow} \distnormal(0,1)
\end{gather*}

\textbf{\em Model 2}:
$n_{X,Y}<100$, $A \equiv \arcsin$

\vspace{-15pt}
\begin{gather*}
z=2\left(A{\sqrt{\hat{p_X}}}-A{\sqrt{\hat{p_Y}}} \right)\sqrt{\dot{n}}
\longrightarrow \distnormal(0,1)
\end{gather*}

\vfill

\end{multicols}

\newpage

\subsubsection{Advanced hypothesis testing}

\begin{multicols}{2}
\paragraph{Parametric test}
is created in a following way:

\begin{enumerate}

  \item Specify the statistical model, i.e. set of assumptions about the sample (examples are given
  on previous page).

  \item Collect data. This step occurs after the 1st, because the model must not depend on data,
  because specifying statistical model on the basis of some properties of the sample leads to
  overestimation of significance level.

  \item Specify the null hypothesis $H$.

  \item Specify the significance level $\alpha$.

  \item 

  \item Calculate the critical region $K_\alpha$.

  \item Make a decision.

\end{enumerate}

\paragraph{Test for goodness-of-fit}
determines if an unknown distribution of interest $F$, given a sample $X$ i.i.d. $F$:

\noindent \textbf{\em Case 1}:
is similar to a given, known distribution, i.e. fits some other distribution

$F_0$ is a specific distribution, $H_0: F=F_0$

\noindent \textbf{\em Case 2}:
belongs to a given, known family of distributions, i.e. fits some family of distributions

$\mathcal{F}_0$ is a family of distributions, $H_0: F \in \mathcal{F}_0$

\paragraph{Test for normality}
is a special case of a goodness-of-fit test, in which we test fitness of our distribution to a normal
distribution (or the whole family of normal distributions).

\paragraph{Test for independence}
determines if two samples are independent.

\end{multicols}

\subsubsection{Selected non-parametric tests}

\begin{multicols}{2}
\paragraph{\underline{Hypothesis tests for goodness-of-fit}} \hspace{0pt}

\vspace{10pt} \noindent \textbf{\em Pearson chi-square test}
$X$ i.i.d. $F=?$, $F_0$ is dist. of interest, $H_0: F=F_0$

$d_i= \frac{N_i - np_i}{\sqrt{np_i}}$,
$p_i = P_{F_0}(X \in S_i) > 0$

\[ T = \Sigma_{i=1}^r d_i^2 \rightarrow_{n \rightarrow \infty} \chi^2_{r-1-t} \]

\vspace{10pt} \noindent \textbf{\em Kolmogorov test} assumes that the sample comes from a continuous distribution.

\[ T = \sup_{x \in \mathbb{R}}{ \left| \hat{F}_n(x) - F_0(x) \right| } \]

\[ T^+ = \max_{i}{ \left( \frac{i}{n} - \hat{F}_0(x_{i:n}) \right) } \]

\[ T^- = \max_{i}{ \left( \hat{F}_0(x_{i:n}) - \frac{i-1}{n} \right) } \]

where $i \in \mathbb{N}$, $i \in [1,n]$ and $x_{i:n}$ is the $i$th sample element taken from sorted sample

\[ T = \max{ \left\{ T^+, T^- \right\} } \]

\paragraph{\underline{Hypothesis tests for normality}} \hspace{0pt}

\vspace{10pt} \noindent \textbf{\em Shapiro-Wilk test}
\begin{gather*}
W = \frac{ \left( \sum_{i=1}^{n} a_i x_{i:n} \right)^2 }{ \sum_{i=1}^{n} \left( x_i - \bar{x} \right)^2 } \\
a_i = \frac{m^T V^{-1}}{ \sqrt{ m^T V^{-1} V^{-1} m } }
\end{gather*}

where $m$ is a vector of expected values of sorted sample from $\distnormal(0,1)$, and $V$ is a covariance matrix of $m$.

Null hypothesis is rejected when $W$ is too small. This test is designed for small sample sizes, $n < 50$. 

\paragraph{\underline{Hypothesis tests for independence}} \hspace{0pt}

\vspace{10pt} \noindent \textbf{\em Wilcoxon signed-rank test}
\begin{gather*}
W = \frac{1}{2}\Sigma_{i=1}^n \bar{r}(Z_i)+\frac{n(n + 1)}{4}
\end{gather*}

\end{multicols}

\newpage

\subsubsection{Linear regression}

\begin{multicols}{2}
\input{cs_8_regression}
\end{multicols}

\newpage

\hspace{0pt}

\newpage

\thispagestyle{empty}

\section{Formulas for the exam}

\section*{Formulas: confidence intervals}
\begin{multicols}{3}
\paragraph{\underline{Conf. intervals for the mean}}
$\mu$ \hspace{0pt} \newline
Conf. interval: $ \left( \mean - k, \mean + k \right) $

\vspace{5pt} \noindent \textbf{\em Model 1}:
$X$ i.i.d. $\distnormal(\mu, \sigma)$, known $\sigma$
\[ k = z_{1-\alpha/2}\frac{\sigma}{\sqrt{n}} \]

\noindent \textbf{\em Model 2}:
$X$ i.i.d. $\distnormal(\mu, \sigma)$, $\sigma=?$
\[ k = \diststudentt^{[n-1]}_{1-\alpha/2}\frac{s_X}{\sqrt{n}} \]

\noindent \textbf{\em Model 3}:
$X$ i.i.d. ?, large $n$
\[ k = z_{1-\alpha/2}\frac{s_X}{\sqrt{n}} \]

\paragraph{\underline{Conf. intervals for variance}}
$\sigma^2$ \hspace{0pt} \newline
or $\var X$ are, in case of first 2 models, defined as:
\[ \left( f(1-\alpha/2), f(\alpha/2) \right) \]
and $f(k)$ is defined for each. The third model has its own definition.

\vspace{5pt} \noindent \textbf{\em Model 1}:
$X$ i.i.d. $\distnormal(\mu, \sigma)$, known $\mu$
\[ f(k) = \frac{n\tilde{s}_X^2}{\distchisquare_{k,n}} \]

\noindent \textbf{\em Model 2}:
$X$ i.i.d. $\distnormal(\mu, \sigma)$, $\mu=?$
\[ f(k) = \frac{(n-1)\tilde{s}_X^2}{\distchisquare_{k,n-1}} \]

\noindent \textbf{\em Model 3}:
$X$ i.i.d. ?, large $n$,

Conf. interval: $ \left( f(-1) , f(1) \right) $
\[ f(k) = \frac{(2n-2)\tilde{s}_X^2}{\sqrt{2n-3} + (k)z_{1-\alpha/2}} \]

\paragraph{\underline{Conf. interval for proportion}}
$p$ \hspace{0pt} \newline
i.e. probability of success.

\vspace{5pt} \noindent \textbf{\em Model 1}:
$X$ i.i.d. $\distbernoulli(p)$, large $n$

Conf. interval: $ \left( \hat{p}-l , \hat{p}+l \right) $
\[ \hat{p} = \frac{1}{n} \sum_{i=1}^n X_i \mbox{ and }
l = z_{1-\frac{\alpha}{2}}\sqrt{\frac{\hat{p}(1-\hat{p})}{n}} \]

\end{multicols}

\section*{Formulas: non-parametric tests}
\begin{multicols}{2}
\paragraph{\underline{Hypothesis tests for goodness-of-fit}} \hspace{0pt}

\vspace{10pt} \noindent \textbf{\em Pearson chi-square test}
$X$ i.i.d. $F=?$, $F_0$ is dist. of interest, $H_0: F=F_0$

$d_i= \frac{N_i - np_i}{\sqrt{np_i}}$,
$p_i = P_{F_0}(X \in S_i) > 0$

\[ T = \Sigma_{i=1}^r d_i^2 \rightarrow_{n \rightarrow \infty} \chi^2_{r-1-t} \]

\vspace{10pt} \noindent \textbf{\em Kolmogorov test} assumes that the sample comes from a continuous distribution.

\[ T = \sup_{x \in \mathbb{R}}{ \left| \hat{F}_n(x) - F_0(x) \right| } \]

\[ T^+ = \max_{i}{ \left( \frac{i}{n} - \hat{F}_0(x_{i:n}) \right) } \]

\[ T^- = \max_{i}{ \left( \hat{F}_0(x_{i:n}) - \frac{i-1}{n} \right) } \]

where $i \in \mathbb{N}$, $i \in [1,n]$ and $x_{i:n}$ is the $i$th sample element taken from sorted sample

\[ T = \max{ \left\{ T^+, T^- \right\} } \]

\paragraph{\underline{Hypothesis tests for normality}} \hspace{0pt}

\vspace{10pt} \noindent \textbf{\em Shapiro-Wilk test}
\begin{gather*}
W = \frac{ \left( \sum_{i=1}^{n} a_i x_{i:n} \right)^2 }{ \sum_{i=1}^{n} \left( x_i - \bar{x} \right)^2 } \\
a_i = \frac{m^T V^{-1}}{ \sqrt{ m^T V^{-1} V^{-1} m } }
\end{gather*}

where $m$ is a vector of expected values of sorted sample from $\distnormal(0,1)$, and $V$ is a covariance matrix of $m$.

Null hypothesis is rejected when $W$ is too small. This test is designed for small sample sizes, $n < 50$. 

\paragraph{\underline{Hypothesis tests for independence}} \hspace{0pt}

\vspace{10pt} \noindent \textbf{\em Wilcoxon signed-rank test}
\begin{gather*}
W = \frac{1}{2}\Sigma_{i=1}^n \bar{r}(Z_i)+\frac{n(n + 1)}{4}
\end{gather*}

\end{multicols}

\newpage

\thispagestyle{empty}

\section*{Formulas: parametric tests}
\begin{multicols}{2}
\paragraph{\underline{Hypothesis tests for the mean}}
$\mu$

\vspace{5pt}
\noindent \textbf{One-sample z-test}
$H_0: \mu=\mu_0$

We use it to check hypothesis that mean $\mu$ is equal to the specific value $\mu_0$. Critical
region is: $ \mathcal{K}_\alpha = (-\infty, k] \cup [k, +\infty) $

\vspace{5pt}
\textbf{\em Model 1}:
$X$ i.i.d. $\distnormal(\mu, \sigma)$, known $\sigma$

\vspace{-15pt}
\begin{gather*}
T=\frac{\bar{X}-\mu_0}{\sigma}\sqrt{n} 
\overset{H_0}{\longrightarrow} \distnormal(0,1)
\end{gather*}

\vspace{-20pt}
\begin{gather*}
k = z_{1-\frac{\alpha}{2}}
\end{gather*}

\textbf{\em Model 2}:
$X$ i.i.d. ?, large $n$

\vspace{-15pt}
\begin{gather*}
T=\frac{\bar{X}-\mu_0}{s_X}\sqrt{n} 
\overset{H_0}{\underset{n \rightarrow \infty}{\longrightarrow}} \distnormal(0,1)
\end{gather*}

\vspace{-15pt}
\begin{gather*}
k = \diststudentt^{[n-1]}_{1-\frac{\alpha}{2}}
\end{gather*}

\noindent \textbf{One-sample t-test}
$X$ i.i.d. $\distnormal(\mu, \sigma)$, $\sigma=?$, $H_0: \mu=\mu_0$

\vspace{-15pt}
\begin{gather*}
z=\frac{\bar{X}-\mu_0}{s_X}\sqrt{n} 
\overset{H_0}{\longrightarrow} \diststudentt^{[n-1]}
\end{gather*}

\vspace{-15pt}
\begin{gather*}
k = z_{1-\frac{\alpha}{2}}
\end{gather*}

\noindent \textbf{Two-sample z-test}

We use it to check a hypothesis that means of two samples are equal. The test can be applied in
following 2 cases:

\vspace{10pt}
\textbf{\em Model 1}:
$X$, $Y$ i.i.d. $\distnormal(\mu_{X,Y}, \sigma_{X,Y})$,

known $\sigma_X$ and $\sigma_Y$, $H_0: \mu_X=\mu_Y$

\vspace{-15pt}
\begin{gather*}
a_I=\frac{\sigma_I^2}{n_I} \;,\;
z=\frac{\bar{X}-\bar{Y}}{ \sqrt{a_X+a_Y} }
\longrightarrow \distnormal(0,1)
\end{gather*}

\textbf{\em Model 2}:
$X$, $Y$ i.i.d. ?, large $n_{X,Y}$,

$H_0: \mu_X=\mu_Y$

\vspace{-15pt}
\begin{gather*}
a_I=\frac{s_X^2}{n_X} \;,\;
z=\frac{\bar{X}-\bar{Y}}{ \sqrt{a_X+a_Y} }
\underset{n \rightarrow \infty}{\longrightarrow} \distnormal(0,1)
\end{gather*}

\noindent \textbf{Two-sample t-test}

\vspace{10pt}
\textbf{\em Model 1}:
$X$, $Y$ i.i.d. $\distnormal(\mu_{X,Y}, \sigma_{X,Y})$, 

$\sigma_X=\sigma_Y=?$, $H_0: \mu_X=\mu_Y$, 
$d = n_X+n_Y-2$, $a_i = (n_i - 1)s_i^2$

\vspace{-15pt}
\begin{gather*}
\dot{s}^2 = \frac{a_X + a_Y}{d} \;,\; 
b_I=\frac{\dot{s}^2}{n_I}
\end{gather*}

\[ t=\frac{\bar{X}-\bar{Y}}{ \sqrt{b_X+b_Y} }
\overset{H_0}{\longrightarrow} \diststudentt^{[d]} \]

\textbf{\em Model 2}:
$X$, $Y$ i.i.d. $N(\mu_{X,Y}, \sigma_{X,Y})$, 

$\sigma_X=?$ and $\sigma_Y=?$, $H_0: \mu_X=\mu_Y$

\vspace{-15pt}
\begin{gather*}
a_i = \frac{s_i^2}{n_i} \;,\;
d = \frac{(a_X + a_Y)^2}{ \frac{a_X^2}{n_X-1}+\frac{a_Y^2}{n_Y-1} }
\end{gather*}

\vspace{-15pt}
\begin{gather*}
t=\frac{\bar{X}-\bar{Y}}{ \sqrt{ a_X + a_Y } }
\overset{H_0}{\longrightarrow} \diststudentt^{[d]}
\end{gather*}

\vfill
\columnbreak

\noindent \textbf{Paired t-test}
$X$, $Y$ i.i.d. $\distnormal(\mu_{X,Y}, \sigma_{X,Y})$, 

$\sigma_X=?$ and $\sigma_Y=?$, $H_0: \mu_X=\mu_Y$, 

$\forall i: (X_i,Y_i)$ are dep., $Z=X-Y$

\vspace{-15pt}
\begin{gather*}
t= \frac{\bar{Z}-0}{s_Z}\sqrt{n} 
\overset{H_0}{\longrightarrow} \diststudentt^{[n-1]}
\end{gather*}

It is usually used when identical samples are taken at two points in time, and we wish to check
whether there are significant differences between them.

\paragraph{\underline{Hypothesis tests for variance}}
$\sigma^2$

\vspace{5pt}
\noindent \textbf{One-sample test}
$X$ i.i.d. $\distnormal(\mu, \sigma)$, $H_0: \sigma^2=\sigma_0^2$

We use this test to test a hypothesis that $\sigma^2$ is equal to a specific value $\sigma_0^2$.
$ \mathcal{K}_\alpha = (0, \distchisquare_{k, \frac{\alpha}{2}}] \cup [\distchisquare_{k, 1-\frac{\alpha}{2}}, +\infty) $

\vspace{5pt}
\textbf{\em Model 1}:
known $\mu$

\vspace{-15pt}
\begin{gather*}
T = \frac{n s_X^2}{\sigma_0^2}
\overset{H_0}{\longrightarrow} \distchisquare_{n-1}
\end{gather*}

\vspace{-15pt}
\begin{gather*}
k = n
\end{gather*}

\textbf{\em Model 2}:
$\mu=?$

\vspace{-15pt}
\begin{gather*}
T = \frac{(n-1)s_X^2}{\sigma_0^2}
\overset{H_0}{\longrightarrow} \distchisquare_{n-1}
\end{gather*}

\vspace{-15pt}
\begin{gather*}
k = n-1
\end{gather*}

\noindent \textbf{F-test}
$X$, $Y$ i.i.d. $\distnormal(\mu_{X,Y}, \sigma_{X,Y})$, 

$\mu_X=?$ and $\mu_Y=?$, $H_0: \sigma_X^2=\sigma_Y^2$

We use this test to test a hypothesis that two variances $\sigma_X^2$ and $\sigma_Y^2$ (for samples
$X$ and $Y$ respectively) are equal.

\vspace{-15pt}
\begin{gather*}
F = \frac{s_X^2}{s_Y^2} \overset{H_0}{\longrightarrow} \distf^{[n_1-1,n_2-2]}
\end{gather*}

\paragraph{\underline{Hypothesis tests for proportion}} $p$

\vspace{5pt}
\noindent \textbf{One-sample z-test}
$X$ i.i.d. $Bern(p)$, $H_0: p=p_0$

We use this test to test a hypothesis that $p$ is equal to a specific value $p_0$. It applies to two
such models:

\vspace{10pt}
\textbf{\em Model 1}:
$n_{X,Y}\geq100$

\vspace{-15pt}
\begin{gather*}
z=\frac{\hat{p}-p_0}{\sqrt{p_0(1-p_0)}}\sqrt{n}
\overset{H_0}{\underset{n \rightarrow \infty}{\longrightarrow}} \distnormal(0,1)
\end{gather*}

\textbf{\em Model 2}:
$n_{X,Y}<100$, $A = \arcsin$

\vspace{-15pt}
\begin{gather*}
z=2\left(A{\sqrt{\hat{p}}}-A{\sqrt{p_0}} \right)\sqrt{n}
\longrightarrow \distnormal(0,1)
\end{gather*}

\noindent \textbf{Two-sample z-test}
$X$, $Y$ i.i.d. $\distbernoulli(p_{X,Y})$, $H_0: p_X=p_Y$

We use this test to test a hypothesis that parameters $p_X$ and $p_Y$ (which are empirical success
rates for samples $X$ and $Y$ respectively) are equal. It applies to two such models:

\vspace{10pt}
\textbf{\em Model 1}:
$n_{X,Y}\geq100$, $ k_I = \Sigma_{i=1}^{n_I} I_i $

$ \dot{p}=\frac{k_X+k_Y}{n_X+n_Y} \mbox{ , } \dot{n}=\frac{n_X n_Y}{n_X+n_Y} $

\vspace{-15pt}
\begin{gather*}
z=\frac{\hat{p_X}-\hat{p_Y}}{\sqrt{\dot{p}(1-\dot{p})}}\sqrt{\dot{n}}
\underset{n_{X,Y} \rightarrow \infty}{\longrightarrow} \distnormal(0,1)
\end{gather*}

\textbf{\em Model 2}:
$n_{X,Y}<100$, $A \equiv \arcsin$

\vspace{-15pt}
\begin{gather*}
z=2\left(A{\sqrt{\hat{p_X}}}-A{\sqrt{\hat{p_Y}}} \right)\sqrt{\dot{n}}
\longrightarrow \distnormal(0,1)
\end{gather*}

\vfill

\end{multicols}

\newpage

\section{Sources, references}

If you find any mistakes, please send message to \url{bysiekm {at} student.mini.pw.edu.pl}

\vspace{10pt} \noindent Thanks to\ldots
\begin{itemize}
  \item P. Grzegorzewski and M. Gągolewski, for this whole theory and formulas
  \item group AP, for many of the formulas
  \item Karo, for notes
  \item lots of other people, for suggestions, corrections, etc.
  \item Wikipedia
\end{itemize}

\end{document}
