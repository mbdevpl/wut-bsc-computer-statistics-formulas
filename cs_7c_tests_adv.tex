\paragraph{Parametric test}
is created in a following way:

\begin{enumerate}

  \item Specify the statistical model, i.e. set of assumptions about the sample (examples are given
  on previous page).

  \item Collect data. This step occurs after the 1st, because the model must not depend on data,
  because specifying statistical model on the basis of some properties of the sample leads to
  overestimation of significance level.

  \item Specify the null hypothesis $H$.

  \item Specify the significance level $\alpha$.

  \item 

  \item Calculate the critical region $K_\alpha$.

  \item Make a decision.

\end{enumerate}

\paragraph{Test for goodness-of-fit}
determines if an unknown distribution of interest $F$, given a sample $X$ i.i.d. $F$:

\noindent \textbf{\em Case 1}:
is similar to a given, known distribution, i.e. fits some other distribution

$F_0$ is a specific distribution, $H_0: F=F_0$

\noindent \textbf{\em Case 2}:
belongs to a given, known family of distributions, i.e. fits some family of distributions

$\mathcal{F}_0$ is a family of distributions, $H_0: F \in \mathcal{F}_0$

\paragraph{Test for normality}
is a special case of a goodness-of-fit test, in which we test fitness of our distribution to a normal
distribution (or the whole family of normal distributions).

\paragraph{Test for independence}
determines if two samples are independent.
